\section{Eigenwerte}
\subsection{Grundlagen}

\begin{karte}{Reduzibel}
    Eine Matrix \( A \in \C^{n,n} \) heißt reduzibel, wenn es eine Permutationsmatrix \(P\)
    gibt, sodass 
    \[ P^T A P = B = \left[ \begin{matrix} A_{11} & A_{12} \\ 0 & A_{22} \end{matrix} \right] \]
    mit quadratischen Matrizen \( A_{11} \in \C^{k,k}, A_{22} \in \C^{n-k,n-k} \) gilt. 
    Andernfalls heißt \( A \) irreduzibel.
\end{karte}

\begin{karte}{Reduzibilitätskriterium}
    Seien \( A \in \K^{n,n}, B\in \K^{k,k}, X \in \K^{n,k} \) 
    mit \( \K = \C \) oder \(\K = \R \) und es gelte 
    \[ AX = XB, \rk X = k. \]
    Dann gibt es eine unitäre Matrix \(Q \in \K^{n,n}\) mit 
    \[ Q^H A Q = T = \left[ \begin{matrix}
        T_{11} & T_{12} \\ 0 & T_{22}
    \end{matrix} \right], \]
    wobei \( \lambda(T_{11}) = \lambda(A) \cap \lambda(B) \).
\end{karte}

\subsection{Normalformen}

\begin{karte}{Schur-Normalform}
    Zu jedem \(A\in \C^{n,n}\) gibt es eine unitäre Matrix \(U \in \C^{n,n}\) so, dass 
    \[ U^H A U = R = D + N, \]
    wobei \(R\) obere Dreiecksform, \(N\) strikte obere Dreiecksform hat und 
    \(D = \diag(\lambda_1, \ldots, \lambda_n)\) mit den Eigenwerten \( \lambda_1, \ldots, \lambda_n \)
    von \(A\) ist. Die Matrix \(U\) kann so gewählt werden, dass die Eigenwerte in beliebiger 
    Reihenfolge in \(D\) auftreten.
\end{karte}

\begin{karte}{Reelle Schur-Normalform}
    Zu jedem \( A\in \R^{n,n} \) gibt es eine reelle Orthogonalmatrix \(Q\in \R^{n,n}\), 
    so dass 
    \[ Q^T A Q = R = D + N,\]
    wobei \(R\) eine reelle obere Blockdreiecksmatrix ist, deren Blockdiagonalmatrix \(D\) 
    aus \(1\times 1\) und \(2\times 2\) Blöcken besteht, bei denen alle \(2\times 2\) Blöcke 
    konjugiert komplexe Eigenwerte haben. \(N\) ist eine strikte obere Dreiecksmatrix.
\end{karte}

\begin{karte}{Normale Matrizen}
    Eine Matrix \(A\in \C^{n,n}\) ist normal genau dann, wenn \(A A^H = A^H A\). 

    Eine Matrix \(A\in \C^{n,n}\) ist normal genau dann, wenn sie unitär diagonalisierbar ist, 
    d. h. es gibt eine unitäre Matrix \(U\in \C^{n,n}\), so dass 
    \[ U^H A U = \diag(\lambda_1, \ldots, \lambda_n). \]
\end{karte}

\begin{karte}{Jordan Normalform}
    Zu jedem \( A \in \C^{n,n} \) gibt es eine nicht singuläre Matrix 
    \(X\in \C^{n,n}\), so dass
    \[ X^{-1} A X = J = \diag(J_{m_1}(\lambda_1), \ldots, J_{m_r}(\lambda_r)), \]
    wobei 
    \[ J_{m_i}(\lambda_i) = \left[ \begin{matrix}
        \lambda_i & 1 & & 0 \\
        & \ddots & \ddots & \\
        & & \ddots & 1 \\
        & & & \lambda_i
    \end{matrix} \right] = \lambda_i I + N \in \C^{m_i,m_i}. \]
    Ist \(m_i > 1\), so nennt man \(\lambda_i\) einen Eigenwert 
    mit Defekt und sagt, dass \(A\) Defekt ist.
\end{karte}

\begin{karte}{Spektralradius}
    Sei \(A\in \C^{n,n}\) mit Spektralradius \(\rho = \rho(A) = \max_{\lambda\in\lambda(A)} \abs{\lambda}\) gegeben.
    Bezeichnen wir mit \( ||\cdot || \) eine beliebige \(p\)-Norm, \(1\leq p \leq \infty\), 
    so ist für \(T\in \C^{n,n}\) nicht singulär 
    \[ ||A||_T = ||T^{-1} A T|| \] 
    die zugehörige \(T\)-Norm definiert und es gilt 
    \begin{enumerate}
        \item \(\rho(A)\leq ||A||\).
        \item Ist \(A\) diagonalisierbar, dann gibt es eine nicht singuläre Matrix 
        \(T\), sodass \( ||A||_T = \rho \).
        \item Zu jedem \(\epsilon > 0 \) gibt es eine nicht singuläre Matrix 
        \(T(\epsilon)\), so dass \(||A||_{T(\epsilon)} \leq \rho + \epsilon\).
    \end{enumerate}
\end{karte}

\subsection{Störungstheorie, Einschließungssätze}
\begin{karte}{Bauer-Fike}
    Sei \(A\in \C^{n,n}\) diagonalisierbar, \( X^{-1} A X = D = \diag(\lambda_1, \ldots, \lambda_n) \)
    und sei \( \mu \in \lambda(A+E), E \in \C^{n,n} \). Dann gilt in jeder \(p\)-Norm 
    \[ \min_{1\leq j\leq n} \abs{\mu - \lambda_j} \leq \kappa(X) ||E||. \]
\end{karte}

\begin{karte}{Gershgorin}
    Es gilt 
    \[ \lambda(A) \subseteq \bigcup_{j=1}^n \mathcal{D}_j, 
    \qquad \mathcal{D}_j = \set{t\in \C: \abs{z - a_{jj}}\leq r_j}, 
    \qquad r_j = \sum_{l=1, l\neq j}^n \abs{a_{jl}}. \]
    Wegen \(\lambda(A) = \lambda(A^T)\) kann man den Schnitt der 
    Gershgorin-Kreise von \(A\) und \(A^T\) betrachten.
\end{karte}

\begin{karte}{Eingenwert Fehlerabschätzung}
    Sei \(\lambda \in \lambda(A)\) ein einfacher Eigenwert von \(A\) 
    und seien \(x, y\) zugehörige rechte und linke Eigenvektoren 
    \[ Ax = \lambda x, y^H A = \lambda y^H. \]

    Dann hat die Matrix \( A + \epsilon E \) für \(\epsilon\) hinreichend 
    klein einen einfachen Eigenwert \( \lambda(\epsilon) \), so dass 
    \[ \lambda(\epsilon) = \lambda + \epsilon \frac{y^H E x}{y^H x} + \mathcal{O}(\epsilon^2). \]
\end{karte}

\begin{karte}{Rayleigh-Quotient}
    Zu gegebener Matrix \(A\) und \(x \neq 0\) heißt 
    \[ \varrho_A(x) = \frac{x^H A x}{x^H x} \]
    Rayleigh-Quotient von \(x\).

    Die Menge 
    \[ \mathcal{F}(A) = \set{\varrho_A(x) : x\in \C^n, x\neq 0} \]
    aller Rayleigh-Quotienten heißt der Wertebereich von \(A\).
    Auch für reelle Matrizen wird der Wertebereich über \(\C^n\) gebildet.
\end{karte}

\begin{karte}{Rayleigh-Quotient Eigenschaften}
    Es gilt 
    \begin{enumerate}
        \item \( \varrho(\gamma x) = \varrho(x) \) für alle \(\gamma \neq 0, \gamma \in \C\).
        \item \( \lambda(A) \subset \mathcal{F}(A) \), d. h. alle Eigenwerte liegen im Wertebereich.
        \item Für normale Matrizen gilt \( \mathcal{F}(A) = \conv(\lambda(A)) \).
    \end{enumerate}
\end{karte}

\begin{karte}{Rayleigh-Ritz}
    Ist \( A\in \C^{n,n} \) hermitesch, dann gilt 
    \begin{enumerate}
        \item \( \lambda_{\min} \leq \varrho(x) \leq \lambda_{\max} \),
        \item \(\lambda_{\max} = \max_{x\neq 0} \varrho(x) \), 
        \item \( \lambda_{\min} = \min_{x\neq 0} \varrho(x) \).
    \end{enumerate}
\end{karte}

\begin{karte}{Bendixon-Hirsch}
    Für \( A\in \C^{n,n} \) gelte für \( \mu_n \leq \mu_1 \) und 
    \( \tau_n \leq \tau_1 \) 
    \[ \lambda \left( \frac{1}{2} (A + A^H) \right) \subset [\mu_n, \mu_1] \quad \text{und} 
    \quad \lambda \left( \frac{1}{2i} (A - A^H) \right) \subset [\tau_n, \tau_1]. \]

    Dann ist \( \mathcal{F}(A) \subset [\mu_n, \mu_1] \times i[\tau_n, \tau_1] \).
\end{karte}

\subsection{Potenzenmethode}

\begin{karte}{TODO}
    Es sei \( A\in \C^{n,n} \) diagonalisierbar mit \( X^{-1} A X = \Lambda = \diag(\lambda_1, \ldots, \lambda_n), 
    X = [x_1 \cdots x_n], ||x_i|| = 1 \) und es gelte 
    \[ \eta := \frac{\abs{\lambda_2}}{\abs{\lambda_1}} < 1. \]
    Ist für \( a := X^{-1}y_0, a = [\alpha_1 \cdots \alpha_n]^T \) die erste Komponente \( \alpha_1 \neq 0 \), 
    dann gilt für \( y_{k+1} = A y_k \):
    \begin{enumerate}
        \item \( y_k = \lambda_1^k [\alpha_1 x_1 + \mathcal{O}(\eta^k)] \) (\(y_k/\lambda_1^k\) konvergiert 
        gegen einen Eigenvektor von \(A\)).
        \item Für die Rayleigh-Quotienten gilt \( \varrho_A(y_k) = \lambda_1 + \mathcal{O}(\eta^k) \).
        \item Falls \(A\) normal ist, gilt \( \varrho_A(y_k) = \lambda_1 + \mathcal{O}(\eta^{2k}) \).
    \end{enumerate}
\end{karte}

\begin{karte}{Potenzenmethode}
    \begin{tabbing}
        \(y_0 \neq 0\) gegebener Startvektor, \(y_0 = y_0 / ||y_0||\) \\
        for \= \( k = 0,1,\ldots \) do \\
        \> \( z_{k+1} = A y_k \) \\
        \> \( \varrho_k = y_k^H z_{k+1} \) \\
        \> \( y_{k+1} = \frac{1}{||z_{k+1}||} z_{k+1}\) \\
        end for
    \end{tabbing}
    Methode ist langsam, wenn \( \eta \) nahe bei eins ist. 
    Man kann außerdem nur den betragsgrößten Eingenwert und zugehörigen 
    Eigenvektor berechnen. 
\end{karte}

\begin{karte}{Inverse Potenzenmethode mit Shift}
    \begin{tabbing}
        \( \mu \in \C \) gegebener Shift, \(y_0 \neq 0\) gegebener Startvektor, \(y_0 = y_0 / ||y_0||\) \\
        Berechne die LR-Zerlegung von \( \mu I - A \)\\
        for \= \( k = 0,1,\ldots \) do \\
        \> Löse \( (\mu I - A) z_{k+1} = y_k \) mit der LR-Zerlegung \\
        \> \( y_{k+1} = \frac{1}{||z_{k+1}||} z_{k+1}\) \\
        \> \( \varrho_{k+1} = \varrho_A(y_{k+1}) = y_{k+1}^H A y_{k+1} \) \\
        end for
    \end{tabbing}
\end{karte}

\begin{karte}{Rayleigh-Quotienten Iteration}
    \begin{tabbing}
        \( \mu_0 \in \C \) gegebener Shift, \(y_0 \neq 0\) gegebener Startvektor, \(y_0 = y_0 / ||y_0||\) \\
        for \= \( k = 0,1,\ldots \) do \\
        \> Löse \( (\mu_k I - A) z_{k+1} = y_k \) \\
        \> \( y_{k+1} = \frac{1}{||z_{k+1}||} z_{k+1}\) \\
        \> \( \mu_{k+1} = \varrho_A(y_{k+1}) = y_{k+1}^H A y_{k+1} \) \\
        end for
    \end{tabbing}
    
    Ist \(A\) normal, dann konvergiert die Folge \( (\mu_k)_k \) der Rayleigh-Quotienten Iteration 
    lokal kubisch gegen einen Eigenwer von \(A\), d. h., wenn die Folge \(y_k\) gegen einen Eigenvektor 
    konvergiert, dann konvergiert \(\mu_k\) gegen den zugehörigen Eigenwert.
\end{karte}