\section{Nichtlineare Gleichungssysteme}
\subsection{Fixpunktiteration}
\begin{karte}{Konvergenzordnung}
    Sei \( (e_k)_k \) eine Folge 
    mit \(e_k > 0 \) und \(e_k \rightarrow 0\).
    Die folge \textit{konvergiert (mindestens) mit Ordnung } \(p \geq 1\), 
    wenn ein \(C > 0\) und ein \(k_0 \in \N_0\) existiert, sodass 
    \[ e_{k+1} \leq C e_k^p \quad \forall k\geq k_0 \]
    gilt. Für \(p = 1\) fordert man zusätzlich \(C < 1\).
\end{karte}